\chapter{\docname}

В данной лекции вводится понимание машинного обучения как задачи эмпирического обобщения, приводится понятие приближения алгоритма к подразумеваемой порождающей модели. Вводятся понятия возможности обучения через притерии информативности выборки. Приводится более подробный перечень моделей обучения, частично, с пояснением их порождающих гипотез и ограничений (регрессионный анализ, методы на основе репрезентативных случаев, алгоритмы регуляризации, алгоритмы с деревом принятия решений, байесовские алгоритмы, кластеризация, метод опорных векторов, ассоциативные правила, нейронные сети, глубокое обучение, уменьшение размерности, ансамблевые алгоритмы).

Цель лекции заключается в более глубоком ознакомлении с применением каждого алгоритма анализа данных и подготовки к их применению с чётким представлением о том, что у каждой модели обучения, которую они реализуют, есть алгоритмические ограничения на сложность изображаемой ими порождающей модели, и о том, какой размен следует учитывать при моделировании эмпирических данных более или менее сложной порождающей моделью.
