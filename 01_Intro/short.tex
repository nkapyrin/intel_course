\chapter{\docname}

В данной лекции даются базовые определения машинного обучения как дисциплины (анализ данных, искусственный интеллект, кэспертная система, алгоритм), вводится типология объектов и признаков, приводится пояснение по разбиению методов машинного обучения (с учителем или без учителя), даются иллюстрации разных устоявшихся постановок задачи машинного обучения, систематизированные по комбинациям входных и выходных переменных (регрессия, ранжирование, классификация, частичное обучение, кластеризация, оценивание плотности).

Даётся описание процесса построения моделей для обучения. Приводится пояснение по формированию функционала качества обучения, вводится понятия обработки данных -- нормализация, фильтрация, борьба с переобучением.

Цель лекции заключается в ознакомлении с машинным обучением как с процессом построения алгоритмов, особым образом рабоющих с данными, с целью аппроксимации некоторой предполагаемой порождающей модели этих данных, в отсутствие её точного описания.
